\documentclass[11pt,a4paper]{article}
\usepackage[utf8x]{inputenc}
\usepackage[T1]{fontenc}
\usepackage{url}
\usepackage{graphicx}
\graphicspath{{figures}}

\usepackage{mathptmx} % Use Times Font

\usepackage{geometry}
 \geometry{
 a4paper,
 total={170mm,257mm},
 left=20mm,
 top=20mm,
 }

\usepackage{bibentry}
\nobibliography*

\title{Cloud Systems AE1 Report}

\author{
  % Just did alphabetical order, but you're free to change if you want
  Alistair Johnston \\  2560836j
  \and
  Stefan Vučković \\ 2621177v
}

\date{\today} % Can do \date{{}} if we want empty

\begin{document}


\flushbottom
\maketitle
\thispagestyle{empty}

\section*{Application context}
The application being considered is `a multiplayer online game by hobbyist game developers for a game jam'.
This application requires:
\begin{enumerate}
\item{Low network latency, the game must be responsive to player input};
\item{Reliable networking, the server should have the lowest rate of packet loss possible};
\item{High network throughput, the server must be able to process many users at once without slowdown}.
\end{enumerate}

For this application the developers are students who have access to the free and student tiers of \href{https://aws.amazon.com/free/}{AWS}, \href{https://azure.microsoft.com/en-us/free/students}{Azure} and \href{https://cloud.google.com/free?hl=en}{GCP}. The developers want to compare these offerings to each other to decide which offering would best support their game.

\section*{Methodology}
For each cloud provider (AWS, Azure, GCP):
\begin{enumerate}
  \item{Use Terraform to provision the VM;} %bit more needed here
  \item{Allow 5 minutes to elapse to mitigate potential warm-up effects;}
  \item{Install dependencies on the VM: \texttt{sudo apt install iperf3};}
  \item{Run \texttt{iperf3 -s} to initialise an iperf server;}
\end{enumerate}

After the IPerf server is initialised the following steps are performed on a local machine (e.g. Laptop, Desktop):
\begin{enumerate}
  \item{Run \texttt{iperf3 -c <IP> -t <duration>  -u -P <parallel\_hosts> --json --logfile <result\_file.json>} on the laptop}
  \item{Run \texttt{ping -c <count> <VM ip>} > <result\_file>}
\end{enumerate}

\section*{Results}
\textbf{Do some cool graphs and stuff}

\begin{figure}
\includegraphics[width=\textwidth]{placeholder.jpg}
\caption{Box and whisker plot showing the round trip times for each VM}
\end{figure}

\begin{figure}
\includegraphics[width=\textwidth]{placeholder.jpg}
\caption{Plot of network throughput on AWS VM}
\end{figure}

\begin{figure}
\includegraphics[width=\textwidth]{placeholder.jpg}
\caption{Plot of network throughput on Azure VM}
\end{figure}

\begin{figure}
\includegraphics[width=\textwidth]{placeholder.jpg}
\caption{Plot of network throughput on GCP VM}
\end{figure}

\textbf{Meant to have 6 graphs from spec, but not sure what other 2 would be of value}

\begin{figure}
\includegraphics[width=\textwidth]{placeholder.jpg}
\caption{Grafana Something?}
\end{figure}

\begin{figure}
\includegraphics[width=\textwidth]{placeholder.jpg}
\caption{Grafana Something?}
\end{figure}

\textbf{Also want to do a stats test to see diff in whatever metrics seem of interest (should be able to do all if we want)}

\section*{Conclusions}

\end{document}
